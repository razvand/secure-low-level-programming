% vim: set tw=78 sts=2 sw=2 ts=8 aw et ai:
\documentclass{training}

% Comentează liniile de mai jos în cazul în care nu există cod de inclus.
%\usepackage{code/highlight}
%\usepackage{color}        % dacă e folosit highlight
%\usepackage{alltt}        % dacă e folosit highlight

\title[Capitol 2]{Capitol 2}
\subtitle{Spații de adrese}
\author[Răzvan]{Răzvan Deaconescu\\razvan.deaconescu@cs.pub.ro}
\date{}

\begin{document}

\frame{\titlepage}

\frame{\tableofcontents}

% NB: Secțiunile nu sunt marcate vizual, ci doar apar în cuprins
\section{Securitatea aplicațiilor}

\begin{frame}{Introducere}
  \begin{itemize}
    \item alfa
    \item beta
    \item Extra
      \begin{itemize}
        \item arhaic
        \item proterozoic
      \end{itemize}
  \end{itemize}
\end{frame}

\begin{frame}{Continuare}
  \begin{itemize}
    \item brad
    \item molid
    \item carpen
  \end{itemize}
\end{frame}

\section{Procese}

\section{Spațiul de adrese al unui proces}

\section{Alocarea și accesarea spațiului de adrese}

\section{Sumar}

\begin{frame}{Concluzii}
  \begin{itemize}
    \item povestea cu limbile
  \end{itemize}
\end{frame}

\begin{frame}{Cuvinte cheie}
  \begin{columns}
    \begin{column}{0.5\textwidth}
      \begin{itemize}
        \item certificări
        \item LPIC
        \item GNU/Linux
        \item Debian
        \item sysfs
        \item procfs
        \item udev
        \item dispozitive
      \end{itemize}
    \end{column}
    \begin{column}{0.5\textwidth}
      \begin{itemize}
        \item apropos, man, info
        \item linie de comandă
        \item shell
        \item biblioteca readline
        \item pachete, PMS
        \item dpkg, apt
        \item utilizatori, grupuri
        \item parole
      \end{itemize}
    \end{column}
  \end{columns}
\end{frame}

\begin{frame}{Resurse utile}
  \begin{itemize}
    \item James Turnbull, Peter Lieverdink, Dennis Matotek -- Pro Linux System Administration
    \item \url{http://elf.cs.pub.ro/pisr/}
    \item \url{http://www.lpi.org/index.php/eng/certification/the_lpic_program}
    \item \url{http://debian.org/doc/user-manuals}
    \item \url{http://wiki.debian.org/}
    \item \url{http://www.debian-administration.org/}
  \end{itemize}
\end{frame}

\end{document}
